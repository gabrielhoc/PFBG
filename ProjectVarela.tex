\documentclass{article}

\usepackage{gensymb}
\usepackage{breakcites}
\usepackage{titling}
\usepackage{authblk}
\usepackage{comment}
\usepackage{amsmath}
\usepackage{caption}

\title{\textbf{Are climate change impacts on lizard genetic diversity mediated by thermal tolerances?}}
\author{Gabriel Henrique de Oliveira Caetano\textsuperscript{1}}
\affil{\textsuperscript{1}Universidade de Bras\'ilia}
\date{\vspace{-5ex}}

\begin{document}
	\pagenumbering{arabic}
	
	\maketitle
	
{\raggedright

\section{Abstract}

\section{Introduction}

\paragraph{} Conservation of genetic diversity is crucial for the persistance of species and populations, as diversity confers resilience to environmental change \cite{frankham2005genetics} and diseases \cite{king2012does}. Genetic diversity is also important for the functioning of ecosystems and the services they provide \cite{faith2010evosystem, srivastava2012phylogenetic, rosauer2016phylogeography}. Understanding how this diversity originates allows us to better conserve the conditions necessary for its maintenance, preserving the evolutionary and biogeographical history contained in the genetic material of species \cite{purvis2000nonrandom, davies2011phylogenetic}.

\paragraph{} The genetic diversity of populations is deeply affected by demographic factors such as population size \cite{kimura1979neutral, leffler2012revisiting, hague2016does} and dispersion\cite{tigano2016genomics}. Those demographic factors are in turn influenced by environmental conditions (cite). This environmental inffluence is modulated by species characteristics, such as physiology \cite{huey1991physiological, walther2002ecological, kearney2009mechanistic}. Environmental temperatures also affect how much time ectothermic animals are active each day \cite{grant1988thermally, adolph1993temperature, sinervo1994growth}, and time of activity restricts the amount of energy individuals can acquire and allocate for growth, maintenance and reproduction \cite{porter1973behavioral, sinervo1994growth, kearney2009mechanistic}.

\paragraph{} The brazilian Cerrado is the most diverse savanna in the world, with a high proportion of endemic species \cite{oliveira2002cerrados}. This diversity is threatened not only by fires, but by deforestation used to make way for pastures and plantations \cite{ratter1997brazilian}, and climate change, which has been linked to population declines and species extirpations all around the globe \cite{bellard2012impacts}. A large percentage of South American biodiversity originated during the Quaternary period (from 2.58 million years ago to the present day), a period with intense climatic changes which deeply affected vegetation dynamics and species diversification in the continent \cite{mayle2004assessment, carnaval2009stability, werneck2011diversification, collevatti2012recovering, werneck2012climatic, ledo2017historical, costa2018biome} but see \cite{melo2016coalescent}. Cycles of expansion and contraction of forested areas isolated enclaves of Cerrado and species associated with those \cite{van1974pleistocene, absy1976some, van1994amazonia, mayle2004assessment}. These enclaves experienced higher climatic instability than core Cerrado areas \cite{werneck2012climatic} and as such, are expected to have lower species diversity and intra-specific genetic diversity than more stable regions, such as core Cerrado areas \cite{werneck2011diversification, carnaval2009stability, carnaval2008historical}. In fact, Cerrado enclaves were found to have relatively poor and unstructured lizard communities, which are indicative of poor demographic performance \cite{gainsbury2003lizard}. 

%This isolation might have led to local adaptations which influenced their current genetic diversity. Current climate change is happening much more rapidly, possibly leading to great loss of biodiversity \cite{sinervo2010erosion}. 

%Vegetation distribution Costa et al 2018

\paragraph{} Species and populations respond differently to temperature changes depending on their thermal tolerance, i.e. the temperature ranges in which they  better perform activities important for their survival and reproduction \cite{porter1973behavioral}. Populations experiencing thermal conditions inside their ideal ranges will have higher reproductive rates \cite{adolph1993temperature}, which leads to increased genetic diversity \cite{zamudio2016phenotypes}. That is especially evident in animals whose metabolism is very influenced by environmental temperatures, such as reptiles and insects. This makes those animals excellent models to study the subject and identify patterns that can be tested in other kinds of organisms.

\paragraph{} My goal is to investigate if the genetic diversity of lizard populations is determined by the interaction of their thermal physiology with temperatures they experienced during their evolutionary history. My hypothesis is that populations that experienced climates inside their thermal tolerances for longer during the Quaternary period achieved higher reproductive performance and consequently, have higher genetic diversity in the present day. I will then investigate what consequences future climate change might bring to the genetic diversity of those animals, using the information learned from the climate changes in the past.

\paragraph{} Therefore, we seek to answer the folloing questions: 

\begin{itemize}
	
\item What are the spatial and temporal scales in which paleoclimatic variabilty affect the current genetic diveristy of tropical lizard populations?

\item Is thermally constrained time of activity determinant of historic demography of tropical lizard populations?

\item Does historical climate instability at ecotones and enclaves make populations in those areas less diverse then those at core Cerrado sites?

\item Will future climate change affect the genetic diversity of tropical lizard populations in core, ecotone and enclave sites differentially?

\end{itemize}

\section{Methods}

\paragraph{} As a part of an ongoing collaboration involving Universidade de Brasília, Instituto Nacional de Pesquisas da Amazônia and University of California Santa Cruz, we have obtained tissue samples and thermal tolerance data for at least 20 species of lizards in the following locations: Brasília, Distrito Federal (15.7998\degree S, 47.8645\degree W), Nova Xavantina (14.6644\degree S, 52.3585\degree W), Gaúcha do Norte (12.9656\degree S, 53.5636\degree W) and Alta Floresta (9.8765\degree S, 56.0855\degree W) at Mato Grosso state; Lagoa da Confusão (10.9201\degree S, 50.1833\degree W) and Pium (10.4428\degree S, 49.1819 \degree W) at Tocantins state; Canindé de São Francisco (9.6419\degree S, 37.7878\degree W) at Sergipe state and Parque Nacional dos Campos Amazônicos (8.4553\degree S, 61.1283\degree W) at Amazonas state. This dataset includes sites in the core of the Cerrado, in the border of the Cerrado with the Amazon forest and on Cerrado enclaves inside the Amazon Forest. I will sample an additional enclave site in Boa Vista (2.8194\degree N, 60.6714\degree W), Roraima state. I will evaluate candidate species from the data set and pick species that occur in the Cerrado core, border and enclaves to perform the study.

\paragraph{} Tissue samples will be used to determine if populations belong to different spatially structured lineages, establish the phylogenetic relationships between them, estimate the time since they diverged, quantify the amount of genetic diversity in each one and estimate the effective population sizes of each lineage at the Mid-Holocene (MH, 6000 years ago), Last Glacial Maximum (LGM, 22000 years ago) and Last Interglacial (LIG, 130000 years ago). The genetic diversity of each population will be quantified through the number of polymorphic sites, number of haplotypes and haplotype diversity of the mitochondrial genes cytochrome-\textit{b} (cyt b) and NADH dehydrogenase subunit 2 (ND2) \cite{santos2014landscape}. Lineages will be identified using the generalized mixed Yule coalescent method \cite{pons2006sequence}. Time of divergence between populations will be estimated using the standard mitochondrial DNA divergence rate for lizards (2\% per million years, \cite{thorpe2005molecular}) with 1\% standard deviation and relaxed molecular clock \cite{santos2014landscape}. Effective population sizes at past times will be estimated using Bayesian Skyline Plots \cite{drummond2005bayesian}, in software BEAST \cite{bouckaert2014beast}. All statistical analysis will be done in program R, unless otherwise specified \cite{r2018}.

% cite packages

\paragraph{} Data on thermal preferences were obtained by placing lizards in MDF (Medium Density Fiberboard) thermal gradients, with 100 cm in length, 15 cm in width e 30 cm in height, open at the top with a 2 cm layer of sand and vermiculite at the bottom. The thermal gradient (approximately from 20 \degree C to 50 \degree C) was established by positioning a 60 W incandescent lamp at one site and an ice pack at the opposite side \cite{paranjpe2013evidence}. Animals free to choose their preferred temperatures for one hour, with temperature sensors attached to their body (type T, 1-mm diameter thermocouples, Omega\textsuperscript{\textregistered} Engineering) connected to a datalogger which records temperatures measured at the sensors every minute (Eltek\textsuperscript{\textregistered} 1000 Series Squirrel Meter Data Logger 64K, 10 Channel 1001WD). The temperature measures during the first ten minutes of the experiment were excluded, since animals were acclimating to the set up \cite{paranjpe2013evidence}. The thermal gradient data will be used to determine the range of preferred temperatures of each population, defined as the central 90\% range of temperatures registered in the datalogger, averaged among all individuals of each population \cite{caetanotime}. I will use a simple ANOVA to test if thermal tolerance limits differ significantly between lineages and between populations inside each lineage. %clarify
For lineages which had not yet originated in any of those past times, the thermal preferences of the most recent common ancestor will be reconstructed through Markov Chain Monte Carlo phylogenetic interpolation \cite{pagel2004bayesian}. All procedures were approved by the ethics in animal use committee of University of Brasília (Process: 33786/2016).

\paragraph{} For all spatial analysis described next, I will use WorldClim \cite{fick2017worldclim} global climate surfaces (minimum, maximum and average monthly temperatures and total monthly precipitation), for the present (1961-1990), MH, LGM, as well as for the years 2050 and 2070 under carbon emission scenarios RCP 4.5 and RCP 8.5. The same variables for LIG will be obtained from \cite{otto2006simulating}. I will also create surfaces of thermally constrained time of activity for each population, considering their thermal tolerances estimated as above, under climate conditions at the present, past and future scenarios described above. For that, I will estimate daily variation in air temperature as a sinusoid ranging from maximum to minimum air temperatures, then account for how much time those temperatures fall inside the species thermal tolerance range \cite{caetanotime, sinervo2010erosion}, using R package Mapinguari \cite{caetanotime}.
\begin{comment}
This is the simplest approach, which seems to work well. We also have operative temperature models and Kearney’s microclim surfaces. I have a paper in review in which I compare the three methods as distribution predictors for a lizard. Sinusoid was equivalent to microclim and much better than operative. Microclim would be ideal, but it is not available for the past or future. I think this could be improved by using the variation estimated in microclim to offset WorldClim past and future temperature projections, but this needs to be validated
\end{comment}
These estimates will be capped by day length at each location, estimated at each site using Corripio’s method \cite{corripio2003vectorial}. I will also calculate a vegetation index using dynamic global vegetation models \cite{sitch2008evaluation} for all present, past and future scenarios described above.

\subsection{Spatial and temporal scales and ecophysiology}

\paragraph{} I will establish circles of 10, 50, 100 and 200 km radius around the point coordinates of focus populations, from which I will extract averages and variances of environmental, ecophysiological and vegetation variables from the present and past surfaces described above. Since temperatures varied fairly unidirectionally between the time steps considered \cite{de2018late}, the rate of change since each time step can be approximated as (\ref{eqn:Eq1}), in which $\delta Pdt$ is the rate of change through time of an environmental variable. $P_{s}$ is the value of the predictor at time step \textit{s}. $N_{t}$ is the number of time steps between time \textit{t} and the present. $D_{t}$ is the total duration between time \textit{t} and the present. 

\begin{equation}
\label{eqn:Eq1}
\delta P_{t} = \frac{\sum\limits_{s=1}^{N_{t}} \lvert P_{s} - P_{s+1} \rvert}{D_{t}}	
\end{equation}

\paragraph{} Rate of change values for each variable will then be used as predictors of present genetic diversity, together with the values for the present as control, in Random Forest models. I will repeat the modeling using values extracted from each circle size and and past time, then compare model accuracy to establish which option generates the best predictors. %clarify
\begin{comment}
I think this approach can give us info on spatial variability while avoiding all the uncertainty of estimating distributions, which would be especially hard to do for individual populations.
\end{comment} 
The Random Forest algorithm was chosen due to its robustness to the high levels of collinearity expected from the choice of predictors \cite{james2013introduction}. I will also include matrices of geographical and phylogenetic distances as covariates to control for those sources of autocorrelation. The accuracy of the models will be assessed through True Skill Statistics (TSS) derived from cross-validation with 30\% of the original data set, which will be set aside before model fitting. The importance of predictors will be evaluated by the average Increase in Node Impurity (INI) resulting from their removal from decision trees \cite{james2013introduction}.



\subsection{Ecotones, enclaves and core sites}

\subsection{Future climate change effects on diversity}

%How to include the temporal effects on future projections?

\paragraph{} I hope the results of this study help guiding efforts to conserve South American biodiversity, by identifying priority areas for conservation, raising awareness of climate change impacts on biodiversity, and hopefully help make the case for the reduction of carbon emissions. I will make the data and statistical analysis used publicly available, to facilitate the replication of this methodology for other regions of the world and other groups of animals and plants.

\bibliography{ProjectVarela}
\bibliographystyle{apalike}

%Genetic diversity is a key variable that allows populations to adapt to global changes. Examples. what is genetic diversity. why genetic diversity is important. how some populations are in danger because they have few genetic diversity, examples.

%Climatic cycles have been pushing species to adapt to the changing conditions. Since 3 million years ago, climate was dominated by glacial-interglacial cycles. blabla. in South America, rainforest and savanas changed their distribution... blabla. examples. In this context, ectotherm species, such as lizards, needed to adapt or perish.

%Genetic diversity is linked to Pleistocene refugia (climatically stable areas are more diverse genetically than instable areas). So, past climatic changes are key factors explaining current genetic diversity.

%But how current ectotherms are going to deal with the ongoing fast temperature warming?

%set 3 main objectives, clearly! be extremely precise about what you want to answer, which data you need to answer it, which methods are you applying, and which results you expect.

% and try to relate all with climate change, and why lizards are important and key species in their ecosystems because they are food for other species, control insects, etc.

}

\end{document}